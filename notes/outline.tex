\documentclass[12pt]{article}
\makeatletter
\def\input@path{{../paper/}}
\makeatother
\usepackage[commenters={OA,AA,DJM}]{shortex} % adjust initials for comments
\usepackage{authblk}
\usepackage[round]{natbib}
\usepackage[margin=2.5cm]{geometry}
\newcommand{\email}[1]{\href{mailto:#1}{#1}}

% minor adjustments to ShorTeX
\let\argmin\relax\DeclareMathOperator*{\argmin}{argmin}
\let\argmax\relax\DeclareMathOperator*{\argmax}{argmax}
\DeclareMathOperator*{\minimize}{minimize}
\DeclareMathOperator*{\maximize}{maximize}
\DeclareMathOperator{\subjto}{\ \text{subject to}\ }
\renewcommand{\top}{\mathsf{T}}
\renewcommand{\d}{\mathsf{d}}
\newcommand{\cond}{\; \m| \;}

\graphicspath{{fig/}}
\setlength{\parindent}{0pt}
%\setlength{\parskip}{11pt}
\usepackage{enumitem}



% -- Begin document --------------------------------------------------------

\title{\huge Supplementing local $R_t$ estimation during periods of geographical dropout}
\author{Draft outline}
\date{Last revised: \today}

\begin{document}
\maketitle
%\RaggedRight % uncomment to enable ragged right

\section{Introduction}

\bitem
\item General intro to Rt estimation
\item Discussion of real-time
\item Emphasis on ``local'' methodologies
\item Extension to estimation across many locations simultaneously 
\item Not all locations have the same reporting frequency
\item High-level description of the methods and a ``killer figure'' that shows
what we do.
\eitem

\section{Methods}

\section*{Bayesian local-global methodology}\label{sec:bayes}

We assume $p\geq 2$ locations (the local level) compose a region (the global level) with random mixing within each location. We do not model interactions among locations.

\bitem
\item Let $I_{t, \ell}$ be incidence in location $1\leq \ell \leq p$ at time $t$
(usually in days).
\item Let $I_t$ be the incidence over the entire region with $I_t = \sum_{\ell}
y_{t, \ell}$. 
\eitem

Consequently, we can employ a renewal model to estimate $R_t$ from time $t=1,\ldots, T + H$ as:
\[
& \mathbb{P}\left(R_t \cond I_1^{t}\right) \text{ given } I_t \sim \text{Noise} 
\left(R_t\Lambda_t\right) \text{ and } \Lambda_t = \sum_{i=1}^{t-1}
\omega_{i}I_{t-i}\label{eq: 1}
\]


$\Lambda_t$ is the total infectiousness in the region, the weights $\omega_{i}$
compose the generation time distribution of the disease and
$\sum_{i=1}^{\infty}\omega_{i} = 1$. `Noise' is some noise distribution of
choice. 

We consider that there is dropout \ie a loss of data, in location $\ell$ from
time $t=T$.

\bitem
\item All locations have the same generation time distribution and $\Lambda_{t,
\ell} =\sum_{i=1}^{t} \omega_{i}I_{t-i, \ell}$.
\item The location with dropout has unknown $I_{\ell,t}$ for $T \leq t \leq T+H$.
\item Local-level renewal models provide the posterior $\mathbb{P}\left(R_{t,
\ell}  \cond I_{1, \ell}^{t}\right)$ only up to $t = T$.
\eitem

There are two options for providing estimates during the dropout time period.
For any time in this period $1 \leq h \leq H$ we can either:
\benum
\item Baseline: project from our last local posterior $\mathbb{P}\left(R_{T, \ell}  \cond I_{1, \ell}^{T}\right)$ to time $T+h$.
\item Correction: borrow information from the global posteriors $\mathbb{P}\left(R_{T+h}  \cond I_{1}^{T+h}\right)$.
\eenum

Our proposed correction uses the last local posterior as a prior and then draws
information from the global posteriors (which exclude $\ell$) to update this
into the posterior:
\[
& \mathbb{P}\left(R_{T+1, \ell} \cond I_{1, \ell}^T, \, \sum_{m \neq \ell} I_{T+1, m} \right) \propto \mathbb{P}\left(R_{T+1, \ell}  \cond \sum_{m \neq \ell} I_{T+1, m}\right)\mathbb{P}\left(R_{T+1, \ell}  \cond I_{1, \ell}^{T}\right)\label{eq: 2}
\]


This follows due to the conditional independence of the infections in locations.
Later time steps are obtained by iterating this process sequentially (with the
prior reset to the dropout posterior from the last time step). We have not yet
implemented this full version and so test a `deterministic dropout correction'
via the following sequence:

Input $\{R_t\}_{t=1}^{T+H}$, $\{R_{t, \ell}\}_{t=1}^{T}$, $\Lambda_{\ell,t}$.
For $h = 1,\ldots,H$, do

\benum
\item Predict $\hat{I}_{T + h, \ell} = R_{T+h}\Lambda_{T+h, \ell}$ (expectations from renewal models).
\item Convolve $\Lambda_{T+h, \ell} = \sum_{i=1}^{T+h-1} \omega_{i}\tilde{I}_{T + h
- i}$, where $\tilde{I}_j = I_j \1[j\leq T] + \hat{I}_j \1[j>T]$.
\item Estimate local $\hat{R}_{T+h, \ell} = \epsilon_{h} R_{T+h} \frac{\Lambda_{T+h, \ell}}
{\Lambda_{T+h+1, \ell}}$, with $\epsilon_h$ as a factor we currently set to 1.
\eenum

\section*{Current and suggested implementations}\label{sec:implement}

The above deterministic algorithm was modified during actual implementation as
follows:

\bitem
\item It is not necessary to sequentially compute $\hat{R}_{T+h, \ell}$ and this
procedure ignores it. 
\item Because location $\ell$ is unavailable for $t=T+1,\ldots,T+H$, we also
cannot compute $\Lambda_t$ for $t=T+2,\ldots,T+H$. For $h \geq 1$ we used the
correction $1 - \Lambda_{\ell, t}/\Lambda_{t}$, to rescale regional incidence
(and convolved incidence).
\item Once we have the sequence $\{\hat{I}_{\ell, t}\}$ for $t=1,\ldots,T+H$, we
simply input this into the local $R_t$ estimation routine as pseudodata.
\eitem

We are predicting $\hat{I}_{T + h, \ell}$ using $R_{T+h}$ and
$\Lambda_{T+h, \ell}$, both of which are smooth. Then we reconvolve and
iterate. It may be more productive to view this as a process with 2 modules, a
\texttt{forecaster} and an \texttt{Rtestimator}. Then the meta procedure is

\benum
\item Use \texttt{forecaster} to produce $\{\Lambda_{T+h, \ell}\}$ for $2\leq h
\leq H$. We do this conditional on all available information (incidence at the
location, the region, or some related quantity). 
\item We can supplement these forecasts with auxiliary signals (wastewater,
deaths etc.). Currently, we have a minimal \texttt{forecaster} (uses global $R_t$
and local convolved incidence).
\item Calculate $\hat{I}_{T+h, \ell}$ by taking the first (backward) differences
of $\{\Lambda_{T+h, \ell}\}$.
\item Use \texttt{Rtestimator} once across the pseudo time series to produce
$\{R_{t, \ell}\}_{t=1}^{T+H}$
\eenum

\section*{Potential paper outline}\label{sec:outline}

This study could combine the algorithm maybe as a small R function that users
can apply to data, or equivalently we can setup the \texttt{forecaster} module
such that it offers outputs that can simply be integrated within user-preferred
\texttt{Rtestimator}modules. There are several questions to investigate that
could complete the paper. 


\bitem
\item What is the value of combining local and global information relative to
projecting forward locally? When should this work?
\item How should we define the global scale? Perhaps pick the areas with a
history of having similarly synched epidemics? Nearest neightbours?
\item Given $p$ locations in a region, how many can we tolerate simulatanous
dropout in? What about other patterns? How long (relative to the generation
time) does dropout need to be before there is no chance of correction?
\item Can we inform the additional parameter $\epsilon_h$ in some way to improve
the correction? Does this require other data e.g., mobility?
\eitem

Practically, we could combine some more simulated examples using a combination
of the \texttt{rtestim} and \texttt{EpiFilter} code that we already have
developed that can provide insights into when and how global information can
supplement local data scarcity as well as prvide guidelines for when this global
information will not work. This could be about 2 figures. Then about 3 figues
with the more impactful content that uses real examples, perhaps from the CFA
pipeline pr the other data sources we discussed. As baselines we can always
compare to local forecasting and hold out real data to use for validation. 

\end{document}

\bibliographystyle{../paper/rss}
\bibliography{../paper/dajmcdon}      
