\documentclass[12pt]{article}
\makeatletter
\def\input@path{{../paper/}}
\makeatother
\usepackage[commenters={OA,AA,DJM}]{shortex} % adjust initials for comments
\usepackage{authblk}
\usepackage[round]{natbib}
\usepackage[margin=2.5cm]{geometry}
\newcommand{\email}[1]{\href{mailto:#1}{#1}}

% minor adjustments to ShorTeX
\let\argmin\relax\DeclareMathOperator*{\argmin}{argmin}
\let\argmax\relax\DeclareMathOperator*{\argmax}{argmax}
\DeclareMathOperator*{\minimize}{minimize}
\DeclareMathOperator*{\maximize}{maximize}
\DeclareMathOperator{\subjto}{\ \text{subject to}\ }
\renewcommand{\top}{\mathsf{T}}
\renewcommand{\d}{\mathsf{d}}

\graphicspath{{fig/}}
\setlength{\parindent}{0pt}
\setlength{\parskip}{11pt}
\usepackage{enumitem}

% -- Begin document --------------------------------------------------------

\title{One-pager on $R_t$ estimation with geographical dropout}
\author{KP, DJM}
\date{Last revised: \today}

\begin{document}
\maketitle
%\RaggedRight % uncomment to enable ragged right


\section{As presented at the conference}\label{sec:proof}

\bitem[nosep]
\item Let $y_{\ell,t}$ be incidence in location $\ell$ at time $t$
\item Let $y_t$ be observed incidence over the region with $y_t = \sum_{\ell}
y_{\ell, t}$ 
\item We have regional estimates of $R_t$ from time $t=1,\ldots,T + H$.
\item There is a subregion $\ell$ with data available only up until time $t$.
\item Define $\Lambda_t = \sum_{i=1}^{t} \omega_{i}y_{t-i}$ to be the convolved
incidence in the region and $\Lambda_{\ell,t}$ be the same for location $\ell$
\eitem

The ``deterministic dropout correction'' is:

Input $\{R_t\}_{t=1}^{T+H}$, $\{R_{\ell, t}\}_{t=1}^{T}$, $\Lambda_{\ell,t}$.\\
For $h = 1,\ldots,H$, do
\benum[nosep]
\item Predict $\hat{y}_{\ell,T + h} = R_{T+h}\Lambda_{\ell,T+h-1}$ (set to it's
expected value)
\item Convolve $\Lambda_{\ell,T+h} = \sum_{i=1}^{T+h} \omega_{i}\tilde{y}_{T + h
- i}$, where $\tilde{y}_j = y_j \1[j\leq T] + \hat{y}_j \1[j>T]$.
\item Estimate local $\hat{R}_{\ell,T+h} = R_{T+h} \frac{\Lambda_{\ell, T+h-1}}
{\Lambda_{\ell, T+h}}$.
\eenum

\section{As actually implemented}

Once we started implementing, a few minor changes were necessary.

\bitem[nosep]
\item We don't actually use $\hat{R}_{\ell,T+h}$ for anything. So, we don't
calculate it. 
\item Because location $\ell$ is unavailable for $t=T+1,\ldots,T+H$, we also
don't have $\Lambda_t$ for $t=T+1,\ldots,T+H$. For $h \geq 1$ we used the
correction $1 - \Lambda_{\ell,t}/\Lambda_{t}$, to rescale regional incidence
(and convolved incidence).
\item Once we have the sequence $\{\hat{y}_{\ell,t}\}$ for $t=1,\ldots,T+H$, we
just throw that into the local $R_t$ estimation routine.
\eitem

\section{Suggested implementation}

We are predicting $\hat{y}_{\ell,T + h}$ using $R_{T+h}$ and
$\Lambda_{\ell,T+h-1}$, both of which are smooth. Then we reconvolve and
iterate. It may be more productive to view this as a process with 2 modules, a
\texttt{forecaster} and an \texttt{Rtestimator}. Then the meta procedure is

\benum[nosep]
\item Use \texttt{forecaster} to produce $\{\Lambda_{\ell, T+h}\}$ for $1\leq h
\leq H$. We do this conditional on all available information (incidence at the
current location, the region, or whatever we want in between). We can even use
auxiliary signals (wastewater or similar). Our current version is just a very
simple \texttt{forecaster} (uses global $R_t$ and local convolved incidence).
\item Calculate $\hat{y}$ by taking the first differences of $\{\Lambda_{\ell, T+h}\}$.
\item Use \texttt{Rtestimator} to produce $\{R_{\ell, t}\}_{t=1}^{T+H}$
\eenum



\end{document}

\bibliographystyle{../paper/rss}
\bibliography{../paper/dajmcdon}      
